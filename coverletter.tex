%!TEX TS-program = xelatex
%!TEX encoding = UTF-8 Unicode
% Awesome CV LaTeX Template for Cover Letter
%
% This template has been downloaded from:
% https://github.com/posquit0/Awesome-CV
%
% Authors:
% Claud D. Park <posquit0.bj@gmail.com>
% Lars Richter <mail@ayeks.de>
%
% Template license:
% CC BY-SA 4.0 (https://creativecommons.org/licenses/by-sa/4.0/)
%


%-------------------------------------------------------------------------------
% CONFIGURATIONS
%-------------------------------------------------------------------------------
% A4 paper size by default, use 'letterpaper' for US letter
\documentclass[11pt, a4paper]{awesome-cv}

% Configure page margins with geometry
\geometry{left=1.4cm, top=.8cm, right=1.4cm, bottom=1.8cm, footskip=.5cm}

% Specify the location of the included fonts
\fontdir[fonts/]

% Color for highlights
% Awesome Colors: awesome-emerald, awesome-skyblue, awesome-red, awesome-pink, awesome-orange
%                 awesome-nephritis, awesome-concrete, awesome-darknight
\colorlet{awesome}{awesome-red}
% Uncomment if you would like to specify your own color
% \definecolor{awesome}{HTML}{CA63A8}

% Colors for text
% Uncomment if you would like to specify your own color
% \definecolor{darktext}{HTML}{414141}
% \definecolor{text}{HTML}{333333}
% \definecolor{graytext}{HTML}{5D5D5D}
% \definecolor{lighttext}{HTML}{999999}

% Set false if you don't want to highlight section with awesome color
\setbool{acvSectionColorHighlight}{true}

% If you would like to change the social information separator from a pipe (|) to something else
\renewcommand{\acvHeaderSocialSep}{\quad\textbar\quad}


%-------------------------------------------------------------------------------
%	PERSONAL INFORMATION
%	Comment any of the lines below if they are not required
%-------------------------------------------------------------------------------
% Available options: circle|rectangle,edge/noedge,left/right
% \photo{./profile.png}
\name{Jonathan}{Schoreels}

\position{Software Engineer{\enskip\cdotp\enskip}Computer Scientist{\enskip\cdotp\enskip}Technical Architecture Enthusiastic}

\mobile{(+32) 474/64.12.88}
\email{jonathan.schoreels@gmail.com}
% \homepage{www.posquit0.com}
\github{JSchoreels}
\linkedin{jonathanschoreels}
% \gitlab{gitlab-id}
\stackoverflow{3833404}{jonathan-schoreels}
\twitter{@JSchoreels}
% \skype{jonathan.schoreels@gmail.com}
% \reddit{reddit-id}
% \extrainfo{extra informations}

% \quote{``Once you stop learning, you start dying."}


%-------------------------------------------------------------------------------
%	LETTER INFORMATION
%	All of the below lines must be filled out
%-------------------------------------------------------------------------------
% The company being applied to
\recipient
  {N-SIDE SA}
  {Boulevard Baudouin 1er, 25\\B-1348 Louvain-La-Neuve\\Belgium}
% The date on the letter, default is the date of compilation
\letterdate{\today}
% The title of the letter
\lettertitle{Application Letter : Software Solution Developer}
% How the letter is opened
\letteropening{Dear Mr. Parlongue,}
% How the letter is closed
\letterclosing{Thank you,}
% Any enclosures with the letter
%\letterenclosure[Attached]{Curriculum Vitae}


%-------------------------------------------------------------------------------
\begin{document}

% Print the header with above personal informations
% Give optional argument to change alignment(C: center, L: left, R: right)
\makecvheader[L]

% Print the footer with 3 arguments(<left>, <center>, <right>)
% Leave any of these blank if they are not needed
\makecvfooter
  {\today}
  {Jonathan Schoreels~~~·~~~Application Letter}
  {}

% Print the title with above letter informations
\makelettertitle

%-------------------------------------------------------------------------------
%	LETTER CONTENT
%-------------------------------------------------------------------------------
\begin{cvletter}

\lettersection{About Me}
I'm a passionate software engineer striving to constantly learn new skills and perfect the existing ones. Since I was a child, my curious mind pushed me towards Sciences and especially towards Physics and Mathematics, but it's with Computer Sciences that I finally fell in love when I had to choose my classes at the University of Mons. Even since the end of my Master Degree, I'm still continuing to read computer sciences books related to architecture, design \& innovative technologies. Combining passion and work is one of the various things that makes me enthusiastic, days after days. 

\paragraph{}
I'm specializing myself in Java technologies mostly due to its Open Source community and frameworks, which is a very important subject for me, as I'm curious to know how things work. I also have an affection for Functional Programming and Lambda calculus, that I've discovered with OCaml and that is more and more popular in JVM languages with Java 8 or Scala. I discovered Scala in Martin Odersky's class ``Functional Programming Principles in Scala'' on Coursera. 

\lettersection{Why N-SIDE?}
As a Computer Scientist, I sometimes feel that I don't use all the knowledge I learned in usual IT opportunities. For example, my skills and my love for Mathematics aren't often used for those. When I saw the different job descriptions available on your website, it instantly triggered my interest. I've applied here to the Software Solution Developer, but I'm confident that I can use more than my software engineering knowledge in the benefits of N-SIDE, such as Graphs and Networks, Computer Vision, Data-Mining \& Combinatorial Optimization. In conclusion, I have a strong belief that working together would be mutually beneficial.

\lettersection{Why Me?}
I'm often described as someone passionate, and I like to describe it as an inner fire that always pushes me further. I'm however aware that passion must be canalized to not backfire, so I take time to cautiously and methodically think about what to learn and how to prioritize my goals. I'm also a quick learner, autodidact and when I acquire knowledge I love to share it in my turn. That would make me a strong asset for N-SIDE since I could ``soak up'' lot of your experts' knowledge and then redistribute it afterward. In conclusion, I would say my confidence isn't about believing to know everything but about being able to learn anything.

\end{cvletter}


%-------------------------------------------------------------------------------
% Print the signature and enclosures with above letter informations
\makeletterclosing

\end{document}
